\documentclass[12pt]{article}

% Custom commands for cleaner and
% easier implementation in the document.

\newcommand{\fig}[4]{%
    \begin{figure}[H]
        \centering
        \includegraphics[width=#3\textwidth]{#4}
        \caption{#1}\label{fig:#2}
    \end{figure}
}

\newcommand{\framedFig}[4]{%
    \begin{figure}[H]
        \centering
        \framebox{\includegraphics[width=#1\textwidth]{#3}}
        \caption{#4}\label{fig:#2}
    \end{figure}
}

% \newcommand{\code}[4]{%
%     \begin{listing}[H]
%         \caption{#2}
%         \label{code:#1}
%         \inputminted[frame=lines]{#3}{\codePath #4}
%     \end{listing}
% }

\newcommand{\tikzFig}[2]{%
    \begin{figure}[H]
        \centering
        \resizebox{1\textwidth}{!}{%
            \input{\tikzPath #1.tex}
        }%
        \caption{#2}\label{fig:#1}
    \end{figure}
}

\newcommand{\texTable}[3]{%
    \table[H]
        \centering
        \caption{#1}\label{tab:#2}
        \input{\tablePath #3.tex}
    \endtable
}

\newcommand*{\fullref}[1]{\textit{\cref{#1} \nameref{#1}}}
\newcommand*{\Fullref}[1]{\textit{\Cref{#1} \nameref{#1}}}


\usepackage[utf8]{inputenc}
\usepackage[swedish]{babel}

\usepackage{pgfpages}

\usepackage{csvsimple}

\newcommand{\notesPath}{./content/notes/}
\graphicspath{{./attachments/figures/images/}}
\newcommand{\tikzPath}{./attachments/figures/tikz/}
\newcommand{\codePath}{./attachments/code/}
\newcommand{\tablePath}{./attachments/tables/}

\graphicspath{{./attachments/figures/images/}}
\newcommand{\tikzPath}{./attachments/figures/tikz/}
\newcommand{\codePath}{./attachments/code/}
\newcommand{\tablePath}{./attachments/tables/}

\title{Planning report}
\author{Linus Sjöbro}
\date{\today}

\begin{document}
\maketitle

\section{Preliminary title}
Positioning in indoor industrial environments and visual data presentation with augmented reality

\section{Background}
LKAB has a smartphone application that will help their mechanics and technicians to show information for a certain type of machinery out in their production plants.
To get real time process data for a machine an onsite OCR code need to be manually scanned in the application.
When the correct code is scanned the application presents the data in plain text.
Due to the harsh and dirty environment in LKAB's mining production plants the application could face difficulties with the reading of these codes.
The manual scanning also means that the user must know the location of the code, which sometimes could be hard to find.
A better way to get the information for a machine is to automatically fetch the process data.
This could be done by knowing the users position in the production plants.
Since this is indoor environments with no access to GPS signal the users placement must be done with other techniques such as WiFi or BLE Beacons.

\section{Purpose}
This thesis will present a solution how to overcome indoor positioning obstacles in environments like LKAB's mining production plants.
The positioning open up to skip the OCR code reading part.
Research done in this area mostly aim for indoor map positioning where the users position is being displayed on a map.
In some cases the indoor position is being translated to real latitude and longitude positions that can be displayed on Google Maps \cite{DevelopmentMobileIndoor2017}.
The research is also mostly made in either office buildings or markets.
That differs from the aim of this project that only need the indoor position at the production plant itself.

\bigskip

Although plain text presentation of the data is a suitable solution, it might cause confusion which machine the data belongst to if a lot of machines is located in a small area.
Automatic data presentation will make the confusion even bigger since the OCR code no longer need to be manually scanned.
With the help of Augmented Reality (AR) the data presentation for a certain machine could be displayed in a visual way.
This could help the users to easier identify which machine the data belongs to.
It could also give a visual presentation of important parts of the data, such as critical alarms.
Furthermore new inovative AR solutions could help the users to perform various complex tasks and repairs.
\cite{SystematicDesignMethod2020} does adopt AR to visually help with the disassembly and assembly of complex mechanical parts.
This AR implementation does give instructions along the way if needed by the user.
AR could also be used in the traning of the staff as mentioned in \cite{AdoptingAugmentedReality2020}.

\bigskip

To be able to verify and create measurable results a proof of cocept (PoC) Apple iOS application will be developed where the choosen positioning technique is being implemented.
The application will also implement AR to show how it could facilitate the usage of the application.

\section{Goals}
With the above purpose a set of goals can be stated;

\begin{enumerate}
  \item Propose a indoor, no GPS, positioning solution adapted for LKAB's specific production plants environments where WiFi-networks has not yet been well deployed. The solution should include;
  \begin{itemize}
    \item How to acheive as high resolution as possible with the propsed technique.
    \item Data on how many devices of the proposed technique that is needed to create the suitable resolution.
  \end{itemize}
\item Investigate and propose how Augmented Reality (AR) in the industry can facilitate for the users when being presented with real time data. This should include;
    \begin{itemize}
      \item How AR interfaces can be designen to help the users.
      \item How the data can be presented in a clear and innovative way.
    \end{itemize}
  \item Develope a proof of concept Apple iOS application that does implement the results from the investigation in above goals. The application should, at least, include;
  \begin{itemize}
    \item Different real time data realative to the users position.
    \item Present the AR result that can help the users to read important data.
  \end{itemize}
\end{enumerate}

\section{Literature review}
A shallow literature review has been done on the areas around non GPS positioning and Augmented Reality in the industry.
This literature review has set the base for the project planning and involning a deeper dive into the literature \cite{AutonomousSmartphoneBasedWiFi2015, AutomaticConstructionRadio2018, WiFiFingerprintIndoor2012, PracticalFingerprintingLocalization2017} about WiFi and BLE Beacon positioning.

\bigskip

The literature \cite{AdoptingAugmentedReality2020, SystematicDesignMethod2020} will be part of the deeper literature review that is testing implementations for AR in industrial environments.


\section{Time plan}
See link for up to date Notion Gantt time plan.

Link: \href{https://www.notion.so/8a79758540934203ab65633360c26e38?v=030857f20c6349c09f2894d37717f17f}{Time plan}

\newpage
\printbibliography
\end{document}
