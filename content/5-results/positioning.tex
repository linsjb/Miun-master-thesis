\section{Positioning}\label{sec:resultPos}
The positioning test is conducted in two steps and the condition for these are being explained in \cref{sec:methodSoftwareTests}.

\bigskip

\Cref{fig:twoVsFour} presents the accuracy of the testing for each group with one fingerprint/point and four fingerprints/point, as well as the overall accuracy.

\fig{Total accuracy per group for one and four mappings}{twoVsFour}{1}{results/twoVsFour}

As seen in \Cref{fig:twoVsFour} the accuracy of four fingerprints (75\%) is higher than one fingerprint (44\%).
These differences is a result on how well the system can predict the location of the device.
With one fingerprint/point the position is sensitive to signal changes, since there is only one mapped fingerprint at each point in the created groups.
The \acrshort{knn} classification also has less data to test on which will make the prediction more uncertain.

\bigskip

With four fingerprints/point the accuracy is higher, since the prediction is not as sensitive to signal fluctuation because there is more fingerprints to test against.
This also means that the \acrshort{knn} classification has more fingerprints to test against, and will therefore be able to make better predictions.


\subsection{Measurement results per point}\label{sec:resultsPosOneFingerprint}
This sub-section presents to which group each measurement has been classified.
Each point is being classified with the majority result from the three test sets, where each set had ten measurements for each point in each group.
In the presented results the measurements for one fingerprint per point and four fingerprints per point is being compared.
See \cref{sec:methodEvaluation} for more information about this.

\bigskip

Each group has its own name as the correct group to be positioned in.
\Cref{fig:bluePointsResult} present the result for the blue group, \Cref{fig:yellowPointsResult} the yellow group and \Cref{fig:brownPointsResult} for the brown group.
The result presented in these figures shows that the group prediction with one fingerprint per point fluctuate a lot.
This has to do with that only one fingerprint is stores for each point.
Because of this the signal is sensitive to fluctuation of the \acrshort{rssi} signals from the beacons and has problems to predict the correct groups.

\bigskip

With four fingerprints the results is more stable and makes more correct predictions since the fluctuation doesn't make as big impact because of more fingerprints for each point.

\fig{Measurement result per point for blue group}{bluePointsResult}{1}{results/pointsBlue}

In \Cref{fig:bluePointsResult} it's shown that the blue group does get predicted most of the time for both one and four fingerprints/point.
It's also shown that the brown group is almost never predicted.
This can be explained by the location of the yellow group, which is closer than the brown group.
Between the area of the blue group and the yellow and brown there was a large concrete wall.
Beacons far away would be more affected by going through this wall than nearby beacons would, which is the case for the yellow group.

\bigskip

In the cases the yellow group is being predicted it can be explained by the beacons placement.
At certain points beacons far away from the blue group, and closer to the yellow groups, has a clear line of sight than nearby beacons.
These beacons does in these case has stronger \acrshort{rssi} values even if they are further away from the point.
This explanation is true for both one and four points since the beacons has the same arrangement.

\bigskip

Why the brown group get predicted at all has its explanation in signal fluctuation.
If the signal of nearby beacons fluctuates a lot it will make a false prediction, that can be of a group further away.

\fig{Measurement result per point for yellow group}{yellowPointsResult}{1}{results/pointsYellow}

In the yellow group the fluctuation is low with the majority calculated groups which is shown in \Cref{fig:yellowPointsResult}.
This is a result of a more open space where the signals is not being reflected by any walls, as well as more beacons are in range with a clear sign of view, which will give a better prediction accuracy.

\bigskip

Since the brown and yellow group is close to each other minor fluctuations in the \acrshort{rssi} signals can cause the \acrshort{knn} classification to predict the wrong group in some cases, as seen in \Cref{fig:yellowPointsResult}.

\fig{Measurement result per point for brown group}{brownPointsResult}{1}{results/pointsBrown}

In the predicted points for the brown group presented in \Cref{fig:brownPointsResult} the result is not as self explaining.
As seen in the figure the blue group get predicted at two points.
This has to do with similar \acrshort{rssi} values of the beacons, where beacons near the blue group, and beacons in or near the brown group has similar values.
A result from this makes the prediction more unstable with the different mapped fingerprints in the groups.

\bigskip

With one fingerprint/point the prediction is only correct in the first point.
This has the same explanation as in the beginning of this sub-section, that the \acrshort{rssi} values can fluctuate so much that the outcome of the prediction changes.
The prediction is stabilizing with four fingerprints/point because there is more data to predict against, and signal fluctuations doesn't have as large impact.
