\chapter{Introduction}\label{introduction}
\pagenumbering{arabic}

\section{Background and problem motivation}
The Swedish mining company LKAB has a smartphone application, named InfoHub, that will help their mechanics and technicians.
This application is capable to shows real time data and document for a chosen type of machinery out in their iron ore production plants.
To choose which machine to show data from a onsite OCR code need to be manually scanned in the application.
Due to the harsh and dirty environment in these production plants the application could face difficulties reading these codes.
The manual scanning also means that the user must know the location of the code, which sometimes could be hard to find.

\bigskip

A better way to get the information for a machine is to automatically fetch the process data when the user is standing nearby.
This could be done by knowing the users position in the production plants and display the corresponding data to the user.
This is happening in indoor environments where there is no access to GPS signals.
This means that the positioning need to be done with other techniques such as WiFi or \acrlong{ble} Beacons.

\bigskip

When a user scan a verifiable code the application will fetch the documents and real time data for the machine that matches the code.
The application gets the data from Microsoft's Azure cloud services that is being uploaded from LKAB's local onsite systems.
Although this solution is suitable in most cases it can cause confusion which machine the data belongs to if many machines is located in the same area.
Automatic presentation of the data can make the confusion even bigger, since the OCR codes no longer need to be manually scanned.

\bigskip

\acrfull{ar} can help to present this data in a better, visual way.
It could also help the users to easier identify which machine the data belongs to and visualise important data events.
For example critical alarms.
\acrshort{ar} could also help the users to perform various complex tasks and repairs with visual help.


\section{Overall aim}
The overall aim in this thesis work can be divided into two parts.

\bigskip

The first parts is to investigate and implement a solution for indoor non GPS positioning for LKAB's specific environments.
The purpose with this is to be able to skip the OCR code scanning step in the application.

\bigskip

The seconds part is how \acrfull{ar} can be implemented in LKAB's industrial environments and facilitate for a better data presentation.
The purpose here is to present and implement a solution that shows how \acrshort{ar} can be used in the application to help the users to read the data from a chosen machine, and how to present which machine to read the data from.


\section{Problem statement}
The problem statement in this thesis work is stated as follow:
How, as accurate as possible, locate handheld devices in indoor environments without any GPS signal and present specific industrial data in a representable and helpful way with \acrlong{ar}?

\section{Knowledge goals}
\begin{enumerate}
  \item \label{knowGoals:pos} Investigate how to achieve a high indoor positioning resolution without GPS techniques in LKAB's production plants environments, where WiFi-networks has not yet been fully deployed.

  \item \label{knowGoals:ar} Investigate how \acrshort{ar} can be used and facilitate for the application users when being presented with real time machine data.
\end{enumerate}

\section{Implementation goals}
\begin{enumerate}
\item \label{implGoals:posPoc} Develop a \acrfull{poc} Apple iOS application that implement the resulting positioning technique from knowledge goal nr. \ref{knowGoals:pos}. This implementation should result in;
\begin{itemize}
\item Measurable result in form of resolution and location accuracy
\item Data on how much hardware that is needed to correctly position the user with the smartphone
\end{itemize}
\item Implement \acrshort{ar} in the \acrshort{poc} application that shows how it can be used in LKAB's everyday activities in their production plants.
The implementation should include measurable comparison results for;
\begin{itemize}
\item Different \acrfull{ui} components
\item Different \acrfull{ux} components
\end{itemize}
\end{enumerate}

\section{Scope}
The focus in the positioning area will be to only position smartphone devices at one floor plan.
The proof of concept system will therefore not take multiple floor plans into consideration.
The system will be tested in a office environment and not in any of LKAB's production plants due to limitations in testing areas.

\bigskip

The scope in the \acrshort{ar}-part will be to make a test implementation with a limited setup of features to show how it can be used to present data and help the users.
The implementation will therefore not be fully implemented with a full features toolbox that can cover every use case.


\section{Outline}
