\chapter{Introduction}\label{sec:intro} \pagenumbering{arabic}

\section{Background and problem motivation}\label{sec:introBackground}
The Swedish mining company LKAB has
a smartphone application, named InfoHub, that will help their mechanics and
technicians.  This application is capable to shows real time data and document
for a chosen type of machinery out in their iron ore production plants.  To
choose which machine to show data from a onsite OCR code need to be manually
scanned in the application.  Due to the harsh and dirty environment in these
production plants the application could face difficulties reading these codes.
The manual scanning also means that the user must know the location of the code,
which sometimes could be hard to find.

\bigskip

A better way to get the information for a machine is to automatically fetch the
process data when the user is standing nearby.  This could be done by knowing
the users position in the production plants and display the corresponding data
to the user.  This is happening in indoor environments where there is no access
to GPS signals.  This means that the positioning need to be done with other
techniques such as WiFi or \acrlong{ble} Beacons.

\bigskip

When a user scan a verifiable code the application will fetch the documents and
real time data for the machine that matches the code.  The application gets the
data from Microsoft's Azure cloud services that is being uploaded from LKAB's
local onsite systems.


\section{Overall aim}\label{sec:introOverallAim}
The overall aim in this thesis work is to investigate and implement a proof of concept solution for indoor non GPS positioning for LKAB's specific needs and environment.
The purpose with the positioning is to be able to skip the manual OCR code scanning step in the InfoHub application.


\section{Problem statement}\label{sec:introProblemStatement}
The problem statement in this thesis work is stated as follow: How, as accurate as possible, locate handheld devices in indoor environments without any GPS signal?


\section{Verifiable goals}\label{sec:goals}
\begin{enumerate}
\item \label{goal:posInvestigation}
Investigate how to achieve a high indoor positioning resolution without GPS
techniques in LKAB's production plants environments, where WiFi-networks has
not yet been fully deployed.

\item \label{goal:poc}
Develop a \acrfull{poc} Apple iOS application that implement the resulting
positioning technique from knowledge goal \ref{goal:posInvestigation}. This
implementation should result in;
\begin{itemize}
\item Measurable result in form of resolution and location accuracy
\item Data on how much hardware that is needed to correctly position the user with the smartphone
\end{itemize}
\end{enumerate}


\section{Scope}
The focus in the positioning area will be to only position smartphone devices at one floor plan.
The proof of concept system will therefore not take multiple floor plans into consideration.
The system will be tested in a office environment and not in any of LKAB's production plants due to
limitations in testing areas.


\section{Outline}
