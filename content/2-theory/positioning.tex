\section{Indoor positioning} \label{sec:theoryIndoorPositioning}
In indoor environments GPS signals is often low or does not exist at all.
This means that other techniques is needed to be able to position a device in these kind of environments.
The positioning can be done in multiple different ways and with different kind of hardware and techniques.
The techniques used to develop indoor positioning in this thesis work is described in this section.



\subsection{Fingerprint}\label{sec:theoryFingerprint}
A signal fingerprint, or position fingerprint, is a popular method to map an array of signal strengths to a specific coordinate or area.
Fingerprint positioning can be used both for indoor and outdoor positioning.
Indoor fingerprint positioning is based on 2D modelling where positioning via WiFi Access Points (AP) is the most popular alternative.
For indoor positioning the fingerprints are often based on \acrshort{rssi} values from the sending devices for every location. 
\cite{LocationFingerprintingInfrastructure2004, IndoorFingerprintPositioning2017}

\bigskip

Fingerprint based localization is not only used in WiFi based systems but can also be used with other hardware, such as \acrshort{ble} Beacons.
It works the on the same way but instead of collecting WiFi \acrshort{rssi} values the values comes from the Beacons instead.
\cite{PracticalFingerprintingLocalization2017} 

\bigskip

A fingerprint positioning system is built up in a two phase process.
An offline and online phase.\cite{IndoorFingerprintPositioning2017} 

\subsubsection{Offline phase}\label{sec:theoryFingerprintOffline}
In the offline phase (also called observation phase) the fingerprints are being collected and mapped to a location in the environment.
The mapped fingerprints are then being stored in a radio map database.
\Cref{fig:fingerprintOfflinePhaseIllustration} shows the flow for collection the \acrshort{rssi} signals, mapping them to a location and storing them in the radio map database.
This is often an time consuming part since all positions need to be manually scanned.\cite{IndoorFingerprintPositioning2017} 


\fig{Fingerprint offline phase flow \cite{IndoorFingerprintPositioning2017} }{fingerprintOfflinePhaseIllustration}{1}{fingerprintOfflinePhase}

\newpage

\subsubsection{Online phase}\label{sec:theoryFingerprintOnline}
The online phase is the part where a user is interacting with the stored fingerprints.
A device is collecting \acrshort{rssi} values from nearby senders in real time and being compared with the fingerprints in the database.
If a match between the real time \acrshort{rssi} values and the radio map occur the location is being yield back and the device can be positioned.\cite{IndoorFingerprintPositioning2017}
\Cref{fig:fingerprintOnlinePhaseIllustration} shows an illustration of the online phase flow.

\fig{Online phase positioning flow\cite{IndoorFingerprintPositioning2017} }{fingerprintOnlinePhaseIllustration}{1.0}{fingerprintOnlinePhase}



\subsection{Received Signal Strength Indication}\label{sec:theoryRssi}
\acrfull{rssi} is 2D model based position technology indication on the signal strength between a sender and receiver.
It's a less complex method than 3D model based technologies such as Time of Arrival, Time Difference of Arrival or Angle of Arrival that require a proper time synchronization between the sender and receiver.
This opens up for easier software implementations and can reduce the complexity of the signal measurement.
The technology became popular because of this but also for its high accuracy, low cost and low power consumption.
\cite{IndoorFingerprintPositioning2017} 

\bigskip

The technology can be divided into three categories;
\begin{itemize}
\item Trilateration
\item Approximate Perception
\item Scene Analysis \cite{IndoorFingerprintPositioning2017} 
\end{itemize}

\subsubsection{Trilateration}\label{sec:theoryRssiTrilateration}
Trilateration is a positioning technique where three or more senders and converts the signals into spatial distances.
This is used to create radius of circles and where all the circles intersect with each other is where the device is located. \cite{IndoorFingerprintPositioning2017} 

\subsubsection{Approximate Perception}\label{sec:theoryRssiApproxPerception}
In approximate perception the signal from the strongest base station is the positioning criterion.
This technique has a low positioning accuracy and to make it more exact researchers has used a cluster of antennas.
This cluster is then placed at a know location and will improve the accuracy.\cite{IndoorFingerprintPositioning2017} 

\subsubsection{Scene Analysis}\label{sec:theoryRssiSceneAnalysis}
Scene analysis is the most used technique for indoor positioning.
It uses fingerprint matching (\cref{sec:theoryFingerprint}) and does not require any information of where the base stations is located.
The positioning is done with the signal strength from multiple base stations at a given reference point.
Together with a fingerprint data structure the reference point will act as a localization point for the positioning system.\cite{IndoorFingerprintPositioning2017} 

\bigskip

Measurements taken with \acrshort{rssi} values can be used to take signal fingerprints to build up a radio map database for indoor localization.\cite{DevelopmentMobileIndoor2017} 



\subsection{K-Nearest Neighbour}\label{sec:theoryKnn}
% Allmänt om KNN
\acrfull{knn} is a machine learning classification algorithm...

\bigskip

% KNN inom positionering
Because of the high performance and low cost of \acrshort{knn} it has been widely used for non GPS positioning, with good results.
\acrshort{knn} is used to compare a real time fingerprint from the users device with the stored fingerprints from the radio map database.
The algorithm starts with choosing the nearest fingerprint neighbours according to a root-mean-square error.
After this it completes the positioning by calculating the weighted average of the k fingerprint data.
\cite{IndoorFingerprintPositioning2017}  

\subsection{Bluetooth Low Energy Beacons}\label{sec:theoryBleBeacons}

