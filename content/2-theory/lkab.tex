\section{LKAB}
Luossavaara-Kiirunavaara AB (LKAB) is a high tech iron ore and mineral mining company with its base in northern Sweden established year 1890.
LKAB is operating in twelve countries over the world with a total of 4300 employees.
Iron ore is their main area which is being mined and processed in Sweden.
Other areas in the company are industrial minerals, drilling systems, rail transport, rockwork services and property management.
In 2019 LKAB produces 27 million tonnes of iron ore products, with a net sales amount of 31 billion SEK. \cite{LKABBrief} 

\bigskip

LKAB stands in front of one of the largest industrial investments ever in Sweden.
An investment of between 10 to 20 billion SEK that extends over more than 20 years and will create between 2000 to 3000 new jobs opportunities.
This investment will result in carbon-dioxide-free iron ore products. \cite{LKABInvestment}

\bigskip

LKAB was the first underground mine with wireless connectivity and today has the largest underground WiFi-network in the world.
With LKAB 5.0 all vehicles and personnel will be connected to these WiFi-networks.
This shows that the company has a high digitalization maturity. \cite{LKABITDevelopment}

\bigskip

% Onödiga detaljer
% After the iron ore has been mined it need to be processed in LKAB's production plants.
% To do this LKAB has a three step process that start with crushing and sorting of the materials to be left only with the iron ore itself \cite{LKABProcessSorting}.
% After this a concentrating step will follow where the ore is being grinded to give a higher purification \cite{LKABProcessConcentration}. After the grinding process the material is being transformed into a floating slurry that is being chemically treated and various additives are mixed into it \cite{LKABProcessConcentration}. 
% The last step in the process is a pelletizing step where the slurry is being dried, formed into pellets and heated to melt the material together \cite{LKABProcessPelletizing}. \cite{LKABProcessing} 

% \bigskip

The productions plants needs constant maintenance and overseeing to be fully functional.
Due to the large number of machines and the complexity of certain machines it's hard to service them without any additional help.
LKAB has developed a Apple iOS application called Infohub that help the employees out in the productions plants to show real time process data and documentation.

