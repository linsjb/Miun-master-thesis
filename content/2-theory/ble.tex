\section{Bluetooth Low Energy}\label{sec:theoryBle}
The wireless technology \acrfull{ble} is part of the core specifications of Bluetooth 4.0 and is also referred as Bluetooth Smart.
\acrshort{ble} focuses on short-range communication where the transmission strength often is measured in dBm.
A closer distance between the receiver and transmitter means a lower dBm-value, which typically range between around -30 dBm to 0 dBm.
\cite{DevelopmentMobileIndoor2017} 

\bigskip

\acrshort{ble} technology is designed to be used where the devices doesn't need to transmit a lot of data.
With this comes a low power consumption and often low cost, which makes \acrshort{ble} a good candidate for a various of different tasks.
\cite{PracticalFingerprintingLocalization2017} 

\bigskip

\acrshort{ble} is operating at 2.4 GHz in the ISM frequency band and is divided into 40 different channels that are spaced 2 MHz apart.
Three of these 40 channels is used for advertisement, which are placed to avoid interference with other existing technologies such as ZigBee and IEEE 802.11 (WiFi).
This entails that the technology can be implemented in environments that has large WiFi or other wireless networks already deployed.
\cite{PracticalFingerprintingLocalization2017} 

\subsection{iBeacons}\label{sec:theoryBleiBeacons}
iBeacon are small battery powered devices developed by Apple that is based on the existing \acrshort{ble} technology.
Using the iBeacon technology together with the Apple iOS platform open up for a variety of opportunities when in comes to implementing position based applications.
Because of the small footprint and low cost of these devices they are easy to deploy in an environment where they can be used together with a smartphone application.
\cite{BluetoothLowEnergy2018} 

\bigskip

The iBeacon is sending out an advertise signal that consists of three parts;
\begin{itemize}
\item 16-byte UUID
\item 2-byte major value
\item 2.byte minor value \cite{GettingStartedIBeacon2014} 
\end{itemize}

The UUID value is a fixed identifier that will identify a set of beacons.
This value needs to be the same for all the beacons that will be used.
The major value is used to identify a large set of beacons with the same UUID, and the minor value is used to identify a specific beacon.
\cite{GettingStartedIBeacon2014}

\bigskip

An example of the iBeacon deployment mentioned in \cite{GettingStartedIBeacon2014} is a global retail store. The UUID will be the same for all stores, meanwhile the major value is used to identify a specific store and the minor value to identify a department in each of the stores.
\cite{GettingStartedIBeacon2014} 

