\chapter{Conclusion}\label{sec:Conclusion}
This chapter presents a summary and conclusion of the projects work.


\section{Project summary}\label{sec:conclusionProjectSummary}
All goals that was set up in the beginning of this project (\cref{sec:introGoals}) has all been met.
Knowledge has been gathered how a \acrfull{ips}, according to goals \ref{goal:posInvestigation}.
Goal \ref{goal:poc} has been fulfilled with the \acrshort{poc} application that was developed, and the sub goals has been answered with performed positioning tests.

\bigskip

The projects statement has been answered with this work.
Firstly the statement is answered in the motivation of the design choices made (\cref{sec:methodSoftwareDesign}) and lastly the \acrshort{poc} application showed how this can be done.

\section{Project validity}\label{sec:conclusionProjectValidity}


\section{Future work}\label{sec:conclusionFutureWork}
\Acrfull{ips} has a bright future and is a technology that will be adopted more and more in the future.
The result the \acrfull{poc} application developed in this thesis work is in its simplest form.
LKAB will probably use this technology and continue to research and test in the are of indoor positioning.
So, this means that a future work on this work is possible and has a chance of be adopted in real production environments.

\bigskip

In the developed application does not take security into consideration between the device and the server.
Here a better security and signing of devices that only belong to LKAB is a large part that need to be implemented.
The positioning itself could also be refined to better and more accurate position the device. 
But to achieve this the application need to be tested in a real environment.

\bigskip

Another part that would need to be refined is the presentation of the created groups.
If the application going to be used the number of groups would be large.
So to have all groups visible at all time would be a bad user experience.
Here the groups should be filtered depending on which plant and building the device is located in.
This could be done with the major value from the iBeacons.

\bigskip

When creation a new group the same problem would occur.
All machines available will make a huge list that would be hard to search in.
Here to the beacon major value could be used to filter out the relevant machines.

\bigskip

When it comes to the server this could also be refined in some places.
How the request is made to the Azure Cosmos Database should be overseen to avoid high bills form Azure.
Also the training of \acrfull{knn} should also be done only when new groups is created.
