\newpage
\chapter*{Abstract}
\addcontentsline{toc}{chapter}{Abstract}
Positioning of mobile phones or other handheld devices in indoor environments is hard because it's often not possible to retrieve a GPS-signal.
Therefore other techniques need to be used for this.
Despite the difficulties with indoor positioning the Swedish mining company LKAB want to do exactly this in  their production plants.
LKAB has developed an Apple iPhone mobile application to maintain real time process data and documents for a machine.
But to retrieve the information an OCR code need to be manually scanned with the application.
Instead of manually scanning these codes LKAB want to develop an \acrlong{ips} that can automatically locate handheld devices in their production plants.
This thesis aimed to create a \acrfull{poc} Apple iOS application that can position devices without GPS-signals.
In the system developed \acrlong{ble} iBeacons is used to transmit data to the application.
From this data \acrlong{rssi} values is being collected and sent off to a server that transform the values into positioning fingerprints.
These fingerprints are used together with the classification algorithms \acrlong{knn} to determine in which, on pre-hand created, group the user is located.
In these created groups there is a defined set of machines that is being presented back to the user.
Test results conducted with the \acrshort{poc} application shows that the implemented system works and gives a positioning accuracy up to 75\%.


\bigskip

\textbf{Keywords:} RSSI, KNN, fingerprints, indoor positioning, Apple, iOS, application, iBeacon, Beacon
