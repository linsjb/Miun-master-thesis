\section{Software implementation}\label{sec:methodSoftwareImplementation}
% S. Robinson et al. mentions in \cite{SecretsSuccessfulSimulation1995} how to build a model in a good way.
% They also mentions that the best approach is to perform the implementation in small iterative steps where the model is being broken down into smaller parts.
In the implementation of the \acrshort{poc} software in this project the work has been conducted with an Agil approach. 
This is a well-known software that has show its success in software development. \cite{DoesAgileWork2015}
Why the Agile method was used is because of the high flexibility in the priority structure.
If something needed to change it was just to re-prioritise the backlog.
This makes the work very easy to manage but also to see the progress of it, because of the constant updated backlog.
This backlog was built with the gathered knowledge described in  \cref{sec:methodProblemDefinition}.
The implementation is being described in \cref{impl}.

\bigskip

% Together with the software design in \cref{sec:methodSoftwareDesign} an implementation was made to fulfil goal \ref{goal:poc} in \cref{sec:introGoals}.
% With the knowledge gathered in \cref{sec:methodProblemDefinition} a backlog with the require software parts could be set up and worked through with an Agile workflow.
% The implementation is being presented at its whole in \cref{sec:impl} below.
