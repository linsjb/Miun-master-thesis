\section{Evaluation}\label{sec:methodEvaluation}
Evaluation of the tests was conducted by measuring which answer the developed \acrshort{ips} was predicting on each of the given points, being presented in \Cref{fig:testSetup} at \cref{sec:methodTestLayout}.
This answer is which group the device is being located in.
The testing does not include an exact point positioning accuracy, since the exact position isn't important because of the systems groups approach.

\bigskip

To reduce \acrshort{rssi} noise and fluctuation when positioning the device, ten measurements was taken at each point in the groups.
With these ten measurements the majority of the predicted groups decided which group the device belongs to.
Each group was also tested three times to further smooth out the results of the prediction.
From this an accuracy, $A$, could be calculated for each group with help of \Cref{eq:evaluation}.
\Cref{tab:eqExplanation} explains the equation variables.
This accuracy presents how well the system can position the device in the correct group.

\begin{equation}\label{eq:evaluation}	
	A = \frac{C_g}{N}
\end{equation}

\texTable{Equation variable explanation}{eqExplanation}{eqExplanation}

The accuracy was being calculated for both of the test parameters presented in \cref{sec:methodTestParameters} and could then be compared with each other for each group, which is being presented in \cref{sec:resultPos}.
This was to see how well different number of fingerprints could affect the prediction.
A total accuracy was also calculated with the results from all three groups to get a combined performance of how well the system worked.

\bigskip

Besides from the prediction accuracy, the majority resulting group was being plotted for each point to show in which group the point did locate the device.
See \cref{sec:resultsPosOneFingerprint}.
