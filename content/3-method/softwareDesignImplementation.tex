\section{Software design and implementation}\label{sec:methodSoftwareDesignImplementation}
In the method framework by S. Robinson et al. in \cite{SecretsSuccessfulSimulation1995} the second phase of a project is to design, build and validate the model, as described in \cref{sec:methodFrameworkBuildTest}. 
To fully fulfil goal \ref{goal:poc} in \cref{sec:introGoals} and to follow the method framework presented in \cref{fig:methodOverview} a software design and implementation was made.
These two phases was split in two to easier keep the two areas apart.


\subsection{Software design}\label{sec:methodSoftwareDesign}
Although software design is a iterative process that can change over time it's  an important step, as mentioned by xx in...
Just as S. Robinson et al. mentions in \cite{SecretsSuccessfulSimulation1995} a structured model makes it easier to build.
In this work the software was designed from the accumulated knowledge based on the research analysis performed in \cref{sec:methodProblemDefinition}.
This design did set the base for the software and which requirements needed to get a working \acrshort{ips}.

\subsection{Software implementation}\label{sec:methodSoftwareImplementation}
S. Robinson et al. mentions in \cite{SecretsSuccessfulSimulation1995} how to build a model in a good way.
They also mentions that the best approach is to perform the implementation in small iterative steps where the model is being broken down into smaller parts.
This describes a Agile work method which is proved to work well in software projects as xx describes in...
With the knowledge gathered in \cref{sec:methodProblemDefinition} a backlog with the require software parts could be set up and worked through in this Agile way.
