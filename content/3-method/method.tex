\chapter{Method}\label{method}
This chapter describes and motivates the chosen methods in this thesis work.

\fig{Overview of the used model}{methodOverview}{1}{methodFlow}

\Cref{fig:methodOverview} is presenting an overview of the used method in this thesis and in which order it was carried out.
This method is based on the method described by S. Robinson et al. in the paper \textit{Secrets of successful simulation projects} \cite{SecretsSuccessfulSimulation1995}.
The paper describes how a simulation based project should be performed for a successful result.
This thesis project is not a simulation project, but the proposed method framework (just called \textit{the framework} from now on) can still be used since it describe general properties and outcomes.

\bigskip

The framework consist of four smaller phases that each of them has their own purpose.
These has been slightly modified and renamed to better suit this thesis.

\bigskip

The framework start with the problem definition.
Here the problem is being identified, experimental factors and reports is being defined, the scope and level of the model is being determined and a project specification is provided.
To be able to make a strong foundation S.Robinson et al. mentions that a clear set of objectives is an important part. 
Without this a project will almost never succeed.
The authors also mentions that discussions with the customer of the project is important and needed to develop a deep understanding of the system to be modelled.
They also says that it's important to know the experimental factors and reports for the project.
With this being set and decided it's easier to give the right inputs so the resulting outputs is expected.
When it comes to determine the scope and level of the model it tells what elements that's required for the model, where the scope is telling what should be included and the level tells the details.
S. Robinson et al. also mention the importance to get the right type of data to succeed with the model. 
A well written project specification is also important since this will tell if the problem is understood by all involved parties.

\bigskip

In the second, model building and testing, phase of the framework the structure and building of the model is performed.
First the model need to be structured. 
This will help to create the best possible model and also help with documentation.
After this the model is being build with the languages and tools decided.
Here the documentation is being updated consistently under the development.
When the model is implemented it need to be validated, which is the final step of this second phase.
Experiments will start only if the model is fully validated and related to the real world system.
S. Robinson et al. also says that it's important to check so the model is able to meet the objectives of the project.

\bigskip

The third phase in the framework is the experimentation.
Here the experiments is first and foremost executed.
S. Robinson et al. mention previous issues when performing various model experiments.
The first problem is the warm-up period, where it's important that the model is realistic.
Often models does not start in a realistic state and need a warm-up period before data is being monitored and collected.
The second problem is to take a decision how long a model should run and the third problem is to decide how many replications to perform of the experiment.
In the last part of the third phase comes the analysing of the result and draw a conclusion.

\bigskip

In the fourth and last phase the project is being completed and implemented.
Here the result need to be communicated to all involved parties.
To round of the project the documentation need to be created and the project need to be reviewed.

\bigskip

The framework has been a solid foundation in this thesis work and the changes modifications will be described in the following sub sections of the method.

\section{Research analysis and problem definition}\label{sec:methodProblemDefinition}
% In the method framework by S. Robinson et al. in the paper \cite{SecretsSuccessfulSimulation1995} a problem definition is the first phase.
% The problem is being identified, experimental inputs and outputs is being defined, the scope and level of the project is being determined and a project specification is provided.
% To be able to make a strong foundation S.Robinson et al. mentions that a clear set of objectives is an important part. 
% Without this a project will almost never succeed.

% \bigskip

% The authors also mentions that discussions with the customer of the project is important and needed to develop a deep understanding of the system to be modelled.
% They also says that it's important to know the experimental factors and reports for the project.
% With this being set and decided it's easier to give the right inputs so the resulting outputs is expected.
% When it comes to determine the scope and level of the model it tells what elements that's required for the model, where the scope is telling what should be included and the level tells the details.

% \bigskip

% S. Robinson et al. also mention the importance to get the right type of data to succeed with the model. 
% A well written project specification is also important since this will tell if the problem is understood by all involved parties.

% \bigskip

To fulfil goal \ref{goal:fieldInvestigation} and \ref{goal:systemDesign} presented in \cref{sec:introGoals} a problem definition, in form of a case study, and a research analysis was performed.

\subsection{Problem definition}\label{sec:methodProblemDefinition}
To succeed with a project a clear understanding of the problem is a large factor.
Discussions with the customer is therefore important to create a deep understanding of the problem.
Without this knowledge a project will almost always fail.
\cite{SecretsSuccessfulSimulation1995}

\bigskip

This project was faced towards the company LKAB and their positioning problem mentioned in \cref{sec:introOverallAim} and \ref{sec:introProblemStatement}.
Meetings was conducted with the company to fully know what they wanted to solve.
This resulted in a requirements specification which can be seen in appendix \ref{appendix:requirements}.

\subsection{Research analysis}\label{sec:methodResearchAnalysis}
The research analysis was performed as a literature review. 
This resulted in knowledge how to implement a \acrshort{ips} and what techniques being used in previous research.
\Cref{sec:theoryRelatedWork} presents the results from the literature review.


\begin{itemize}

\item Groups
    \begin{itemize}
    \item require lower accuracy
    \item One group can contain many machines
    \end{itemize}
\item fingerprints
\begin{itemize}
    \item Based on multiple signals
\end{itemize}
\item hardware, BLE Beacons as transmitters
    \begin{itemize}
    \item Because of limitations in Apple iOS
    \item only one AP at the time
    \item beacon network is more flexible when testing
    \end{itemize}

\item Backend, handle the positioning
\item classification, knn
\end{itemize}


\chapter{Implementation}\label{impl}
This chapter describes the details about the development of the \acrfull{poc} application that's being implemented to meet requirement number \ref{req:developeApp} presented in appendix \ref{appendix:requirements} and to fulfil goal \ref{goal:poc} (\cref{sec:introGoals}).

\section{Architecture}\label{sec:implArchitecture}
To fully understand the implementation of this project some architectural descriptions is needed.
This section will present different techniques and architectures used in this implementation.


\subsection{Machine}\label{sec:implArchitectureMachine}
A machine is being defines as some kind of equipment places around a industrial hall in an manufacturing environment.
This could for example be a waterpump or electrical engine that drives a stone crusher.
A machine in this work it only a set of data collections in a database and not a real physical machine.


\subsection{Group}\label{sec:implArchitectureGroup}
A group in the system is what's being located with the positioning.
The system sets up invisible boundaries in an undefined spaced out in a production hall.
When a device is being located by the system a group will be yield back instead of a physical exact position.
Machines are tied to these created groups where the machines data will be shown when the group is positioned.


\subsection{Device}\label{sec:implArchitectureDevice}
A device refer to a mobile hand held device that's in this work is an Apple iPhone 12 pro.
This device is running the developed application.


\subsection{Beacons}\label{sec:implArchitectureBeacons}
A beacon, also mentioned as iBeacon is the transmitting hardware used to position a device with the developed system.


\subsection{Server}\label{sec:implArchitectureServer}
All calculations are being taken care of by a server.
This server is not being hosted to any cloud service or a physical server in a data centre, instead it is running on a local computer where the device will communicate with the server only in the same LAN.


\subsection{Frontend vs. backend}\label{sec:implArchitectureFrontBack}
In the developed \acrshort{ips} there are two major parts, a front- and backend.
The application is defined as the frontend of the system since this is the part that a user will interact with and get data presented to it. 
The backend is the developed server software that takes care of all calculations and data predictions, and delivers this to the frontend.


\subsection{API}\label{sec:implArchitectureApi}
An API is the data connection between a device and a server.
Data it sent from the device to the server via specific API \textit{endpoints} that is being defined in the application and created in the server.
The developed API works with two different types of requests, POST and GET.

\bigskip

In a POST request the device is giving some kind of data to the server that will be processed.
When the server is done processing a response is being yield back to the device with either some resulting data, or just a success message.

\bigskip

In a GET request the server is only asked to give data back to the device without any data being sent to the server.
When these different endpoints is being called from the device the server will perform various processes depending on how the API is developed.


\section{Apple iOS application}\label{sec:implApp}
The Apple iOS application was developed in the native iOS platform language Swift, made by Apple \cite{SwiftOrg}.
% A design overview of the application data flow can be seen in \cref{fig:appDesign}.
% The following sub-sections describes the different parts in the application and its functions related to \cref{fig:appDesign}.
To meet requirements \ref{req:createGroups}, \ref{req:positionDevice} and \ref{req:displayData} presented in appendix \ref{appendix:requirements} the application has two main parts.
First is the implementation of how a new group is created followed by the implementation of how to position a device and display the position data.
% \fig{Application data flow overview}{appDesign}{0.5}{applicationDesign}

\subsection{Create a new group}\label{sec:implAppNewGroup}
\fig{New group creation flow chart}{appNewGroup}{1.0}{flowCharts/appNewGroup}

Point \ref{req:createGroups} in the requirement specification (appendix \ref{appendix:requirements}) states that one functionality of the application is to create new groups.
When a new group is being created a set of machines should be selectable and different \acrshort{rssi} values mapped to it.
\Cref{fig:appNewGroup} show the flow chart for the creation of a new group.


\subsubsection{Get machines}\label{sec:implAppNewGroupGetMachines}
To be able to choose machines for a new group they first need to be grabbed from an Azure Cosmos database container \cite{IntroductionAzureCosmos}.
The server is taking care of the database handling, which means the application will get the machines from the server with a HTTPs GET request \cite{GETHTTPMDN}.
The API endpoint that is being called is \textit{.../api/machines/all}.
In the application these request is handled with the iOS API \textit{ResourceUrl} \cite{ResourceURLAppleDeveloper}.

\bigskip

On a successful request all machines are loaded into the application.
These machines is then being presented to the user in a list.
The entries in the list is selectable via a touch action, where the selected machines is mapped to the group.
A machine can be used in several groups, so the machines' IDs are stored in the group. 

\subsubsection{Set group name}\label{sec:implAppNewGroupSetName}
For a group to be easier identifiable by a user, a name is set for it.
In the application a text field is filled with the name and then being sent to the server with the rest of the data.


\subsubsection{Map position}\label{sec:implAppNewGroupMapPos}
A group is the part of the system that is being identified as a position in the undefined space and need position fingerprints tied to it.
Nearby \acrshort{ble} iBeacon signals is picked up in the application with the iOS \textit{CoreLocation} API \cite{CoreLocationApple}.


\bigskip

Since the fingerprints that being used by the server are based on three \acrshort{rssi} values, a criterion is set where at least three iBeacons must be in range to be able to map a new position to the group.
If this condition is not met a warning modal is being showed to the user and the data will not be collected.
When the condition is met the nearby iBeacons is collected and the data is being cleaned.
This because the only important value from the iBeacons is their \acrshort{rssi} and minor values.
\Cref{listing:beaconFingerprint} present an example of the formatted iBeacon data.

\fileListing{Formatted iBeacon fingerprint}{beaconFingerprint}{json}{fingerprint.json}

The position mapping is being done manually within the application's user interface.
When a new position is being mapped the cleaned and collected iBeacons data presented in \Cref{listing:beaconFingerprint} is inserted in a 2D array.
In \Cref{listing:groupFingerprints} an example of this 2D array with different iBeacon values is being presented.
\fileListing{A set of collected fingerprints for a group}{groupFingerprints}{json}{groupFingerprints.json}

\subsubsection{Send data to server}\label{sec:implAppnewGroupSend}
When a name is set, machines selected and iBeacons data mapped the group is ready and can be sent to the server as a \acrfull{json} object \cite{IntroducingJSON}.
This is done against the server API endpoint \textit{.../api/group/create} with a HTTPs POST request \cite{POSTHTTPMDN}, where the iOS \textit{ResourceUrl} API \cite{ResourceURLAppleDeveloper} takes care of the request in the application.
Appendix \ref{appendix:newGroupData} presents an example on a full data example of a new group.

\subsection{Display machines for a set position}\label{sec:implAppSetPos}
This section describes how the application takes the current position and display its mapped data, which fulfil the project requirements \ref{req:positionDevice} and \ref{req:displayData} (appendix \ref{appendix:requirements}).
\Cref{fig:clientGetMachines} presents a flow chart that shows how the device is being positioned and corresponding data displayed.

\fig{HTTPs post request flow chart how to get machines from server based on the current position}{clientGetMachines}{1.0}{flowCharts/appGetMachines}


\subsubsection{Reading of nearby beacons}\label{sec:implAppSetPosReadBeacons}
As illustrated in \Cref{fig:clientGetMachines} the first part to get the machines for the current position is to read the nearby iBeacons.
This is done with help of the iOS API \textit{CoreLocation} \cite{CoreLocationApple}.
The application constant read all nearby iBeacons, but its a user decision when to map against a position, which is done manually.

\bigskip

When mapping the location the application first controls if there is at least three iBeacons in range, since the mapped fingerprints for a group consist of data from three iBeacons.
If the condition is not met a modal will be shown to the user and the scanning for nearby iBeacon will continue.
If the condition is met all the data for the nearby iBeacons will be collected and cleaned.
In this cleaning step the \acrshort{rssi} and minor values is being stored in a \acrshort{json} formatted array, just like \Cref{listing:beaconFingerprint} described in \cref{sec:implAppNewGroup} above.


\subsubsection{Send and retrieve data}\label{sec:implAppSetPosSendRetreiveData}
When the data is cleaned it will be sent to the server with a HTTPs POST request controlled with the iOS API \textit{ResourceUrl} \cite{ResourceURLAppleDeveloper}, where the \acrshort{json} data is stored in the body of the request.
The server is processing the data, which is explained in \cref{sec:implServerSetPos} and then returning the machines for the position in the response message of the HTTPs POST request.


\subsubsection{Data presentation}\label{sec:implAppSetPosShowData}
For the data that is being returned in the HTTPs POST response it's tied to the group where the current fingerprint best match the stores ones.
This data is then being presented to the user.


\section{Server}\label{sec:implServer}

\subsection{Server parts}\label{sec:implServerParts}


\subsection{Server functionalities}\label{sec:implServerFunc}







\input{content/3-method/tests.tex}
