\section{Research analysis and problem definition}\label{sec:methodProblemDefinition}
In the method framework by S. Robinson et al. in the paper \cite{SecretsSuccessfulSimulation1995} a problem definition is the first phase.
The problem is being identified, experimental inputs and outputs is being defined, the scope and level of the project is being determined and a project specification is provided.
% To be able to make a strong foundation S.Robinson et al. mentions that a clear set of objectives is an important part. 
% Without this a project will almost never succeed.

% \bigskip

% The authors also mentions that discussions with the customer of the project is important and needed to develop a deep understanding of the system to be modelled.
% They also says that it's important to know the experimental factors and reports for the project.
% With this being set and decided it's easier to give the right inputs so the resulting outputs is expected.
% When it comes to determine the scope and level of the model it tells what elements that's required for the model, where the scope is telling what should be included and the level tells the details.

% \bigskip

% S. Robinson et al. also mention the importance to get the right type of data to succeed with the model. 
% A well written project specification is also important since this will tell if the problem is understood by all involved parties.

\bigskip

To be able to fulfil goal \ref{goal:posInvestigation} in \cref{sec:introGoals} a problem definition, in form of a case study, and a research analysis was performed.
This is the first part of the method framework (see \cref{fig:methodOverview}) which is based on \cite{SecretsSuccessfulSimulation1995}.


\subsection{Problem definition}\label{sec:methodProblemDefinition}
As S. Ribonson et al. mentions in the paper \cite{SecretsSuccessfulSimulation1995} a clear understanding of the problem that a project will solve is crucial.
They also mentions that discussions with the project customer is important to be able to create a deep understanding of the problem.
Without this a project will almost always fail.

\bigskip

With this in mind a good problem definition was gathered from the company to fully understand the problem and what they wanted with the work.
From this the foundation of the thesis could be set, which is being presented in \cref{sec:introOverallAim} and \ref{sec:introProblemStatement}.
A requirements specification could also be created together with the company, which is being presented in appendix \ref{appendix:requirements}.

\subsection{Research analysis}\label{sec:methodResearchAnalysis}
The research analysis was performed as a literature review. 
This resulted in knowledge how to implement a \acrshort{ips} and what techniques being used in previous research.
\Cref{sec:theoryRelatedWork} presents the result of this literature review.
