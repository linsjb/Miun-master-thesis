\section{Software design}\label{sec:methodSoftwareDesign}
Although software design is a iterative process that can change over time it's  an important step, as mentioned by xx in...
Just as S. Robinson et al. mentions in \cite{SecretsSuccessfulSimulation1995} it's beneficial to first design the software before implementing it.
\bigskip

With the information from S. Robinson et al. and to follow the presented method framework (see \cref{fig:methodOverview}) a software design phase was performed before an implementation was made.

\bigskip

The software was designed based on two factors.
The accumulated knowledge how to design a \acrshort{ips} based on the research analysis performed in \cref{sec:methodProblemDefinition}.
The requirements from the research analysis in \cref{sec:methodResearchAnalysis}.
See appendix \ref{appendix:requirements} for the resulting requirements specification.



\subsection{Design motivation}\label{sec:methodDesignMotivation}
Motivation of the design choices is being presented in this sub-section.


\subsubsection{Classification method}\label{sec:methodSoftwareDesignClassification}
In the previous research that was conducted in this thesis, three different types of classification methods has been used to determine the location of a device.
\acrshort{knn}, \acrfull{svm} and \acrfull{ann}.

\bigskip

In the paper \cite{ComprehensiveStudyBluetooth2013} by C. Gurrin et al. all three methods is used to determine the location in a \acrshort{ips}.
\acrshort{ann} shows a good result but takes some time to train.
\acrshort{svm} has a good accuracy and precision but takes a very long time to train, which makes it impractical to use because of this.
\acrshort{knn} shows the best results over all these three algorithms and also takes the shortest time to train.
Therefore \acrshort{knn} is the best choice based on the conclusion by C. Gurrin et al.

\bigskip

The papers \cite{BluetoothLowEnergy2018, DevelopmentSmartphoneBasedUniversity2021} uses \acrshort{knn} to classify the position and \cite{PracticalFingerprintingLocalization2017} uses a \acrshort{knn}-like classification.

\bigskip

With this information the choice fell to use \acrshort{knn} as the classification algorithm.

\subsubsection{Fingerprint structure}\label{sec:methodSoftwareDesignFingerprint}
In the conducted literature study all papers used \acrshort{rssi} based fingerprints in some way.
\cite{DevelopmentSmartphoneBasedUniversity2021, IndoorPositioningSystem2010} used WiFi \acrshort{ap}´s \acrshort{rssi} values.
\cite{BluetoothLowEnergy2018, PracticalFingerprintingLocalization2017, ComprehensiveStudyBluetooth2013} used the \acrshort{rssi} values from \acrshort{ble} beacons to create the fingerprints.
This indicated that \acrshort{rssi} values are a good value to use when creating a \acrshort{ips}.
\bigskip

Since \acrshort{rssi} based fingerprints was used in all the papers it was also used in this thesis.


\subsubsection{Mobile platform}\label{sec:methodSoftwareDesignMobilePlatform}
The choice of mobile platform to develop the \acrlong{poc} application was Apple iPhone.
This because the company the case study was being performed at (LKAB) uses iPhone's in their business.

\subsubsection{Data calculations}\label{sec:methodSoftwareDesignData}
The calculation of the data and the determination of the position can be done in two different ways.
Either directly on the smartphone or at a external server. 
Both of these comes with their pros and cons.

\bigskip

In the paper \cite{DevelopmentSmartphoneBasedUniversity2021} by A. Leb et al. they discuss these two alternatives.
If the computations is never leaving the system, i.e that smartphone, a higher security could be maintained. 
The drawback with this is a higher power usage with a higher battery drainage since the smartphone will handle all the data calculations.

\bigskip

The opposite solution is to send the positioning data to a server instead and let it take care of the positioning determination.
This will save battery on the devices but lower the security. 
In a server based approach the client need to trust the system.
Another positive factor with a server solution is that all business logic for the system is being moved to the server.
This opens up for a more flexible system where other platforms could be integrated with ease.

\bigskip

Since on device calculations might be the best in some cases the choice fell on a server based approach in this thesis project since the pros outweigh the cons.


\subsubsection{Transmitters}\label{sec:methodSoftwareDesignTransmitters}
The transmitting technique chose in this work was \acrshort{ble} beacons where the beacons had support for Apple's iBeacon protocol (see \cref{sec:theoryBleiBeacons}).

\bigskip

This is the desired choice of two different reasons.
The first reason is limitations in Apple's iOS platform.
To be able to scan for nearby WiFi \acrfull{ap} a locked API is needed \cite{NEHotspotHelperAppleDeveloper}.
Apple only give permission to this locked API in special application cases which being presented in \cite{TechnicalQA1942IOS}.
One example where Apple will allow the usage of it, is if the application is made to help the users to find nearby hotspots by scanning the surrounding area.
This makes is impossible to fetch information from other \acrshort{ap}´s than the one connected to the device, which can only be one at a time.
With this limitation the transmitting choice fell on \acrshort{ble} beacons since these can be scanned with the Apple CoreLocation API.

\bigskip

The other reason was the availability to a solid and large enough WiFi-network to perform the software tests (\cref{sec:methodSoftwareTests}).
A \acrshort{ble} Beacon network is easy and fairly cheep to implement and has a large flexibility on how to position the beacons.
WiFi \acrshort{ap}'s is often permanent mounted and is therefore hard to remove.
It's also hard to expand the network if needed.


\subsection{Resulting software design}\label{sec:methodSoftwareDesignResult}
From the design choices being presented in above section and the requirement specification (see \ref{appendix:requirements} ) a design for the \acrshort{poc} application could be made.
This design is being presented in \cref{fig:softwareDesign} below.


\fig{Resulting software design based on the problem definition and research analysis}{softwareDesign}{1.0}{softwareDesign}


