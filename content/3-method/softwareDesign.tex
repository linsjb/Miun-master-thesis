\section{Software design}\label{sec:methodSoftwareDesign}
Although software design is a iterative process that can change over time it's  an important step to conduct before the development starts. \cite{EngineeringDesignIts1989, ImportanceBusinessProcess2008}

\bigskip

When the software is being designed before implemented obvious obstacles could be avoided and therefore save time, which is important in these kind of projects.
So this was an important part to get right from the beginning because of the time frame for the project.

\bigskip

The software was designed based on two factors, the accumulated knowledge how to design a \acrshort{ips} based on the research analysis performed in \cref{sec:methodProblemDefinition}, and the requirements from the research analysis presented in \cref{sec:theoryRelatedWork}.
See appendix \ref{appendix:requirements} for the resulting requirements specification.


\subsection{Design motivation}\label{sec:methodDesignMotivation}
Motivation of the design choices is being presented in this sub-section.


\subsubsection{Classification method}\label{sec:methodSoftwareDesignClassification}
The classification algorithms used to position the device in this work fell on \acrshort{knn} because it showed the best result in comparisons to \acrshort{ann} and \acrshort{svm}, mentioned in \cref{sec:theoryRelatedWork}.
It's also easier to implement and together with the benefits it was the best choice.


\subsubsection{Fingerprint structure}\label{sec:methodSoftwareDesignFingerprint}
In the conducted literature study presented in \cref{sec:theoryRelatedWork} all the reviewed papers used \acrfull{rssi} based fingerprints to position the devices.
This together with it's high accuracy in an \acrshort{ips} the choice in this project fell on \acrshort{rssi} fingerprints.


\subsubsection{Mobile platform}\label{sec:methodSoftwareDesignMobilePlatform}
The choice of mobile platform to develop the \acrlong{poc} application was Apple iPhone.
This because the company the case study was being performed at (LKAB) uses iPhone's in their business.


\subsubsection{Database}\label{sec:methodSoftwareDesignDatabase}
The database used in the \acrshort{poc} application in the Cosmos Database by Azure \cite{IntroductionAzureCosmos}.
This is a two part motivation.

\bigskip

First because LKAB use Azure to their current Infohub application described in \cref{sec:introBackground}.
So to mimic their current architecture the choice fell on an Azure based database.

\bigskip

Secondly, Cosmos, which is a NoSQL database, was used to save time since a table structure nor table relations needed to be taken into consideration.
This was well suited in this project since both the relation between the data and what each data collection was going to include was hard to decide pre-hand.


\subsubsection{Data processing}\label{sec:methodSoftwareDesignData}
The calculation of the data and the determination of the position can be done in two different ways.
Either in a centralized manner where all the data is being processed directly on the device, or at a decentralized server.
As being presented in \cref{sec:theoryRelatedWork} different approached has been taken in different \acrshort{ips} implementations.

\bigskip

In this thesis the choice fell for a decentralized strategy that use a server to perform all data processing.
This opens up for a more flexible system since the server takes care of the data and the device just send data to, and retrieve data from the server.


\subsubsection{Transmitters}\label{sec:methodSoftwareDesignTransmitters}
The transmitting technique chose in this work was \acrshort{ble} beacons where the beacons had support for Apple's iBeacon protocol (see \cref{sec:theoryBleiBeacons}).
This choice depended on two different factors.

\bigskip

The first reason is limitations in Apple's iOS platform.
A developed application can only read data from the WiFi the device is connected to.
Reading of all nearby WiFi's can only be done if the application has access to a closed internal iOS API \cite{NEHotspotHelperAppleDeveloper}.
To gain access to this API Apple need to be contacted for permission.
Access is only given for application in special cases, such as the application will help a user to find nearby WiFi hotspots \cite{TechnicalQA1942IOS}.
This makes is impossible to fetch information from other WiFi \acrshort{ap}´s than the one connected to the device, which can only be one at a time.
With this hardware limitation the transmitting choice fell on \acrshort{ble} beacons since these can be scanned without any special permission with the iOS API \textit{CoreLocation}.

\bigskip

The other reason was the availability to a solid and large enough WiFi-network to perform the software tests (\cref{sec:methodSoftwareTests}).
A \acrshort{ble} Beacon network is easy and fairly cheep to implement and has a large flexibility on how to position the beacons.
WiFi \acrshort{ap}'s is often permanent mounted and is therefore hard to remove.
It's also hard to expand the network if needed.
