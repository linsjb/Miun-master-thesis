\section{Software tests}\label{sec:methodSoftwareTests}
The third phase in the framework by S. Ribonson et al. in \cite{SecretsSuccessfulSimulation1995} is the experimentation.
Here the experiments is first and foremost executed and then being reviewed.
S. Robinson et al. mention previous issues when performing various model experiments and mean that it's important to think about how the tests should be performed.
This also applies to a project of this kind.

\bigskip

In this thesis project the experimentation phase from S. Ribonson et al. can be translated into software tests, which can be seen in \cref{fig:methodOverview}.
Sometimes the tests might show that the software need some changes before it can be conducted at full power or give the right result.
In these cases the tests leads back to the software implementation to correct or change the software.
After this new tests is being carried out.

\bigskip

The first part of of testing was to design the tests that included beacons placement and test parameters.


\subsection{Hardware placement}\label{sec:methodTestHardware}
The beacons used in the test was the RadBeacon Dot by Radius Networks \cite{RadBeaconDotDatasheet}.
Specification of these beacons can be seen in \cref{tab:radbeaconSpecs} below.

\texTable{RadBeacon Dot specification \cite{RadBeaconDotDatasheet}}{radbeaconSpecs}{radbeacon}

\bigskip

The beacons was arranged in two rows where the first row of four beacons was places 4 meters apart,  and the second row with three beacons was spaced 5 meters apart.
The two rows was spaced 6 meters apart.
The beacons arrangement can be seen in \cref{fig:beaconsPlacement}.

\fig{Beacons placement}{beaconsPlacement}{1.0}{beaconsPlacement}


\subsection{Test parameters}\label{sec:methodTestParameters}
The following parameters was set to be tested;

\begin{itemize}
\item Number of $k$ neighbours in \acrshort{knn} classification
\item Number of mapped fingerprints per group
\end{itemize}
