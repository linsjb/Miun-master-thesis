\section{Software tests}\label{sec:methodSoftwareTests}
Tests were conducted to investigate and measure how well the implemented system could position a device in a set test environment.
This environment was supposed to be a real-world industrial hall, but due to limited test areas it was instead performed in an apartment that tried to mimic the environment.
See \cref{sec:methodTestLayout} for details about the test layout.

\bigskip

Apart from a defined test layout, some kind of variance was needed to see how well the system behaves to changes. The different test parameters is being described in \cref{sec:methodTestParameters} below.
If only one test case was conducted, the test wouldn't tell how well the actual \acrshort{ips} performs.
Simply because it would not be known if it's working or not.
To be able to test the system transmitting hardware was needed to be placed in the test layout.
In \cref{sec:methodTestHardware} the used hardware is being described and introduced.


\subsection{Hardware}\label{sec:methodTestHardware}
The beacons used in the test was the RadBeacon Dot by Radius Networks \cite{RadBeaconDotDatasheet}.
Specification of these beacons can be seen in \Cref{tab:radbeaconSpecs} below, and \Cref{tab:beaconConfig} presents the used configuration.

\texTable{RadBeacon Dot specification \cite{RadBeaconDotDatasheet}}{radbeaconSpecs}{radbeacon}

\texTable{Configuration of used beacons}{beaconConfig}{beaconsConfig}


\subsection{Test layout}\label{sec:methodTestLayout}
In a real-world scenario a broad signal coverage of the industrial hall would be required to be able to position devices around the whole area.
This would be easiest to fulfil if the beacons were placed at both the longest walls of the hall and as high as possible.
The idea is to cover the whole hall with a signal, and to make it equally strong closest on the opposite side.
A stronger signal means fewer beacons to place out which in turn means less hardware and a lower cost.

\bigskip

The apartment used in testing was not large enough to conduct a full-scale test and the beacons was therefore places closer to each other than they would normally be to maximize the area.
The number of used beacons opened up for more signal variance in the layout, which in turn creates fingerprints based on different beacons for each group.
The height for the beacons in the test was set to around 1 meter.

\bigskip

The beacons was arranged in two rows where the first row of four beacons was placed 4 meters apart, and the second row with three beacons was spaced 5 meters apart.
The two rows was spaced 6 meters apart.

\bigskip


Three groups, blue, brown and yellow, was set to be tested where each of the groups had nine points each to be mapped for the fingerprint offline phase.
The blue and yellow groups was 3 meters by 2.5 meters and the brown group was 3 meters by 3 meters.
\Cref{fig:testSetup} presents the test setup with the placement of the iBeacons as well as the three groups.
The numbers represent the point name for each group.

\fig{Test setup with iBeacons placement and areas to test against}{testSetup}{1}{testSetup}

Another layout where the beacons were placed around the marked groups was considered.
A layout like this would benefit in a real scenario if the groups is created before the hardware is being placed.
This would create a wider signal coverage around the created groups.
But, it would require more hardware to cover the whole hall since more obstacles would weaken the signals.
Due to the limited space in testing this approach was not the chosen one since it wouldn't make any difference of the number of beacons in range, because of the small area.

\newpage

\subsection{Test parameters}\label{sec:methodTestParameters}
The following parameters was set to be tested;

\begin{itemize}
\item Number of mapped offline fingerprints per location point
	\begin{itemize}
		\item 1 fingerprints per point
		\item 4 fingerprints per point
	\end{itemize}
\end{itemize}

These parameters were being tested to see which effect the number of mapped fingerprints for each point had on the created groups.
With one fingerprint the mapping for each point would only be done in one direction.
Here the human body can interfere the signal from the beacons and give a lower accuracy while testing the system.
A result from this means that the position of the user can make the prediction harder, if the user is not position in the same way when creating the groups.
With four fingerprints per point the fingerprints would be scanned 360 degrees around the point. 
This prevents the human body to interfere the signal in a negative way and therefore increase the positioning accuracy.
It would also give more fingerprints to predict against.
