\subsection{Mapping a position}\label{sec:implServerSetPos}
A position is set when a call is made to the server HTTPs POST API endpoint \textit{.../api/machines/}.
\Cref{fig:serverSetPositionFlowChart} present the process of mapping a position and returning the machines.

\fig{Flow chart of the process to set a position and return the corresponding machines}{serverSetPositionFlowChart}{1.0}{flowCharts/serverSetPosition}

\subsubsection{Data processing}\label{sec:implServerSetPosDataProcessing}
As mentioned in \cref{sec:implAppSetPos} the application collects all nearby iBeacon signals.
These signals is being sent to the server when above mentioned API endpoint is called, where the body of the request contain the signals data.
Before this data can be used to classify the position with help of \acrfull{knn} it need to be prepared.
In this preparation the data is being transformed to match the style of the fingerprints being stored in the created groups.
\Cref{listing:deviceRawBeaconData} show the raw data that is being delivered from the application and \Cref{listing:processedBeaconData} show the processed data.

\fileListing{Raw collected iBeacons data from the application}{deviceRawBeaconData}{json}{fingerprint.json}

\fileListing{Processed data to match group finterprints}{processedBeaconData}{json}{singleFormatedFingerprint.json}

In the processed data presented in \Cref{listing:processedBeaconData} above the major value of the iBeacon with the strongest \acrshort{rssi} value is saved followed by the three strongest \acrshort{rssi} values.


\subsubsection{KNN training}\label{sec:implServerSetPosKnnTraining}
\fig{KNN training flow chart}{knnTrainingFlowChart}{1.0}{flowCharts/serverTrainKnn}

Before \acrshort{knn} can predict any groups it needs to be trained with the correct type of data.
This data is every mapped fingerprints in every created group.
In \Cref{fig:knnTrainingFlowChart} the process of training the \acrshort{knn} classifier with the correct data is presented.

\bigskip

First the data is being collected form the Azure Cosmos Database \cite{IntroductionAzureCosmos} group container. 
After this the data is prepared so it can be used in the training.
For every fingerprint the corresponding group ID and the fingerprint itself is being stored in an array, illustrated in \Cref{fig:illustratedKnnTrainingData}.

\fig{Illustration of KNN training data}{illustratedKnnTrainingData}{0.2}{knnTrainingDataIllustration}

This data is then being divided in two different arrays, one with only the IDs and one with all the fingerprints.
In this format \acrshort{knn} can be trained with the fingerprints and the true values is stored in the other array.


\subsubsection{KNN classification}\label{sec:implServerSetPosKnnClassification}
When the data is processed and in the same format as the group fingerprints it can be used in a \acrshort{knn} algorithm.
\acrshort{knn} is the classification that does the actual decision in which group the device is located.

\bigskip

When the \acrshort{knn} has predicted in which group the device is located it yields back the ID of the group. 

\subsubsection{Filter machines}\label{sec:implServerSetPosFilterMachines}
First all machines are being collected from a Azure Cosmos Database container \cite{IntroductionAzureCosmos}.
After this all machines are being filtered with the resulting ID from the \acrshort{knn} prediction.

\subsubsection{Response data}\label{sec:implServerSetPosResponse}
When all machines is filtered a collection of the data is being sent back to the application, which is being done in the HTTPs POST response.
The answer is taken care of by the application which is being described in \cref{sec:implAppSetPos}.

