\subsection{Display machines for a set position}\label{sec:implAppSetPos}
This section describes how the application takes the current position and display its mapped data, which fulfil the project requirements \ref{req:positionDevice} and \ref{req:displayData} (appendix \ref{appendix:requirements}).
\Cref{fig:clientGetMachines} presents a flow chart that shows how the device is being positioned and corresponding data displayed.

\fig{HTTPs post request flow chart how to get machines from server based on the current position}{clientGetMachines}{1.0}{flowCharts/appGetMachines}


\subsubsection{Reading of nearby beacons}\label{sec:implAppSetPosReadBeacons}
As illustrated in \cref{fig:clientGetMachines} the first part to get the machines for the current position is to read the nearby iBeacons.
This is done with help of the iOS API \textit{CoreLocation} \cite{CoreLocationApple}.
The application constant read all nearby iBeacons, but its a user decision when to map against a position, which is done manually.

\bigskip

When mapping the location the application first controls if there is at least three iBeacons in range, since the mapped fingerprints for a group consist of data from three iBeacons.
If the condition is not met a modal will be shown to the user and the scanning for nearby iBeacon will continue.
If the condition is met all the data for the nearby iBeacons will be collected and cleaned.
In this cleaning step the \acrshort{rssi} and minor values is being stored in a \acrshort{json} formatted array, just like \cref{listing:beaconFingerprint} described in \cref{sec:implAppNewGroup} above.


\subsubsection{Send and retrieve data}\label{sec:implAppSetPosSendRetreiveData}
When the data is cleaned it will be sent to the server with a HTTPs POST request controlled with the iOS API \textit{ResourceUrl} \cite{ResourceURLAppleDeveloper}, where the \acrshort{json} data is stored in the body of the request.
The server is processing the data, which is explained in \cref{sec:implServerSetPos} and then returning the machines for the position in the response message of the HTTPs POST request.


\subsubsection{Data presentation}\label{sec:implAppSetPosShowData}
For the data that is being returned in the HTTPs POST response it's tied to the group where the current fingerprint best match the stores ones.
This data is then being presented to the user.
