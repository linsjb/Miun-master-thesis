\chapter{Conclusion}\label{conclusion}
% This chapter presents a summary and conclusion of the projects work that start with a project summary.
% This is followed by the validity of the project and ends with some thought about the future work for this project.

\mt{An \acrfull{ips} has been implemented to test what can be achieved for a positioning system where GPS cannot be used.
This thesis project has shown that a fingerprint based \acrshort{ips} with \acrlong{ble} beacon transmitters can be used to positioning a handheld device in an indoor environment.
The positioning accuracy is acceptable and shows that a system like this can be deployed in a real industrial environment.}


\section{Project summary}\label{sec:conclusionProjectSummary}
Knowledge has been gathered how an \acrshort{ips} works, according to goals \ref{goal:fieldInvestigation}, which made the goal fulfilled.
With this knowledge an \acrshort{ips} could be designed to meet goal \ref{goal:systemDesign} which then would be implemented.

\bigskip

To be able to meet goal \ref{goal:poc} a working \acrshort{ips} needed to be implemented in a \acrfull{poc} Apple iOS application which has been done.
This goal has therefore been met successfully.

\bigskip

To see how well the implemented \acrshort{ips} and the \acrshort{poc} system performed, it needed to be tested.
With this test the performance could be evaluated, which means that goal \ref{goal:systemEvaluation} has been met.
This means that all goals in this thesis project have been fulfilled with a good result.

\bigskip

The project's statement that was set at the beginning of the project has also been able to be fulfilled.
First the statement is answered in the motivation of the design choices made (\cref{sec:methodSoftwareDesign}) and last the \acrshort{poc} application implemented in \cref{impl} showed how this was done.


\section{Project validity}\label{sec:conclusionProjectValidity}
The validity of the project can be set to class two out of a total of three.
So, it's at an intermediate level with the following explanation.

\bigskip

Due to the way the tests were performed the results from them can give a pointing finger how the implemented system works.
Fully knowledge about this would be to test the system in a real industrial environment, that has both  larger areas and industrial obstacles.
Only then could the accuracy and the real result be seen in its full picture.

\bigskip

When it comes to the implemented system it could be implemented in a real production scenario with some more work, which is being described in future work below.


\section{Future work}\label{sec:conclusionFutureWork}
\Acrshort{ips}'s has a bright future and is a technology that will be adopted more and more in the future.
The result of the \acrshort{poc} system developed in this thesis work is in its simplest form.
LKAB will probably use this technology and continue to research and test in the area of indoor positioning.
So, this means that a future work on this project is possible and has a chance to be adopted in real production environments.

\bigskip

\mt{In the developed system security is not taken into consideration between the device and the backend server.}
Here a better security and signing of devices that only belong to LKAB is a large part that need to be implemented.
The positioning itself could also be refined to better and more accurate position the device. 
To achieve this the software first need to be tested in a real environment to evaluate how to continue the development.

\bigskip

Another part that would need to be refined is the presentation of the created groups.
\mt{If the system is going to be used, the number of groups would be large.}
So, to have all groups visible at all times would be a bad user experience.
Here the groups should be filtered depending on which plant and building the device is located in.
This could be done with the major value from the iBeacons.

\bigskip

The same problem would occur when creating a new group.
All machines available will make a huge list that would be hard to search in.
Here the beacon major value could also be used to filter out the relevant machines.

\bigskip

When it comes to the backend server, it could be further developed.
One thing that would need to be overseen is how requests is made to the Azure database, since each server request trigger a database query.
This would end up in high bills since Azure is charging for every query.
How \acrfull{knn} is tested would also need to be changed.
\acrshort{knn} is being tested on each server request to the API endpoint that predict the device location.
Instead, it should only be tested when new groups are being created, or already existing ones is being changed.

