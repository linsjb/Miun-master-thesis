\subsection{Mapping a position}\label{sec:implServerSetPos}
A position is set when a call is made to the server HTTPs POST API endpoint \textit{.../api/machines/}.
\Cref{fig:serverSetPositionFlowChart} present the process of mapping a position and returning the machines.

\fig{Flow chart of the process to set a position and return the corresponding machines}{serverSetPositionFlowChart}{1.0}{flowCharts/serverSetPosition}

\subsubsection{Data processing}\label{sec:implServerSetPosDataProcessing}
\mt{The collected \acrshort{rssi} values from the device is being sent to the server via the above mentioned API endpoint.}
Before this data can be used to classify the position with help of \acrfull{knn}, it needs to be prepared.
In this preparation the data is being transformed to match the style of the offline stored fingerprints.
\Cref{listing:deviceRawBeaconData} show the raw data that is being delivered from the application and \Cref{listing:processedBeaconData} show the processed data.

\fileListing{Raw collected iBeacons data from the application}{deviceRawBeaconData}{json}{posBeaconsData.json}

\fileListing{Precessed data to match offline fingerprints}{processedBeaconData}{json}{posSingleFormatedFingerprint.json}

In this processed data presented in \Cref{listing:processedBeaconData} above the major value of the iBeacon with the strongest \acrshort{rssi} value is saved followed by the three strongest \acrshort{rssi} values.

\bigskip

This formatted data is now the online fingerprints that will be classified with the help of the offline fingerprints stored in the groups.


\subsubsection{KNN testing}\label{sec:implServerSetPosKnnTesting}
\fig{K-NN testing flow chart}{knnTestingFlowChart}{1.0}{flowCharts/serverTrainKnn}

\Acrshort{knn} need to test on already existing data before it can predict any groups. 
This data is all mapped fingerprints in every created group.
In \Cref{fig:knnTestingFlowChart} the process of testing a \acrshort{knn} classifier with the correct data is presented.

\bigskip

First the data is being collected from the Azure Cosmos Database group container. 
After this the data is prepared, so it can be used in the \acrshort{knn} testing.
For every fingerprint the corresponding group ID and the fingerprint itself is being stored in an array, illustrated in \Cref{fig:illustratedKnnTrainingData}.

\fig{Illustration of K-NN test data}{illustratedKnnTrainingData}{0.2}{knnTrainingDataIllustration}

After the data has been collected, it's divided in two separate arrays.
One with only the IDs and one with all the fingerprints.
In this format \acrshort{knn} can be tested with the fingerprints and matched with their true groups.


\subsubsection{KNN classification}\label{sec:implServerSetPosKnnClassification}
When the data is processed, according to the data preparation step above, and are in the fingerprint form it can be used to predict the position of the device.
The prediction is made with the \acrshort{knn} classifier that control the online fingerprint with the three nearest offline fingerprints, that will say $k=3$.
From the classification, the offline fingerprints that is closest to the online fingerprint will set which group the device is being predicted to.
The prediction then returns the ID of the group.

\subsubsection{Filter machines}\label{sec:implServerSetPosFilterMachines}
First, all machines are being collected from an Azure Cosmos Database container.
After this, all machines are being filtered on the resulting group ID from the \acrshort{knn} prediction.

\subsubsection{Response data}\label{sec:implServerSetPosResponse}
When all machines are filtered a collection of the data is being sent back to the application, which is  done in the HTTPs POST response.
The application then takes care of the data presentation to the user. 

