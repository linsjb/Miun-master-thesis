\section{Software design}\label{sec:resultSoftwareDesign}
This section present and shortly explain the resulting software design.


\subsection{Group creation}\label{sec:resultsSoftwareDesignGroup}
\Cref{fig:newGroupSoftwareDesign} present the data flow how a new group is created.
As explained in \cref{sec:implAppNewGroup}, a new group is created in the iOS application.
\mt{In this new group, a set of chosen machines is selected as well as a name.}
In the application a number of data points based on iBeacons data is mapped against the group.
\mt{These data points are, in the server, converted to fingerprints and stored in an Azure Cosmos Database.}

\fig{Resulting software design to create a new group}{newGroupSoftwareDesign}{0.7}{softwareDesignNewGroup}

\bigskip

\mt{Details about the implementations for a new group is being presented in \cref{sec:implAppNewGroup} and \ref{sec:implServerGroupCreation}.}


\subsection{Show machines for position}\label{sec:resultsSoftwareDesignPos}
\mt{When a set of groups is created and mapped in an environment, the developed iOS application can be used to position the device.
In \Cref{fig:mapPositionSoftwareDesign} the data flow how a device are being positioned is presented.}

\bigskip

\mt{The Apple iOS application collects the nearby iBeacon signals and send these to the server for further processing.
In the server, the location is being determined from the collected signals with the help of \acrshort{knn} classification.
From the classification, a predicted group (of which the device is probably located in) is used to filter out which machines to send back to the device.
These machines are then retrieved by the application and is presented to the user.}

\fig{Resulting software design to position a device in a group}{mapPositionSoftwareDesign}{0.8}{softwareDesignMapPos}

\mt{More details and explanations of the implementation is described in \cref{sec:implAppSetPos} and \ref{sec:implServerSetPos}.}
