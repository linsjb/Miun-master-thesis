\section{Positioning}\label{sec:resultPos}
\mt{The positioning test is conducted for three different groups, blue, yellow and brown, that was set to be created.
Each of these groups has a total of nine fixed points.
The groups are being created with two different numbers of mapped fingerprints per point.
First one fingerprint and later four fingerprints per each of the nine points.
The created groups are then used to position in which group the device is being located in.
For each point in each group ten measurements are taken when the device is being positioned.
Each group is then tested three times.
The result for each point  is the majority predicted group from the ten measurements and three test runs.}



\subsection{Measurement results per point}\label{sec:resultsPosOneFingerprint}
\mt{This sub-section presents to which group each point measurement has been classified.
In the presented results the measurements for one fingerprint and four fingerprints per point is being compared.}


\subsubsection{\mt{Blue group}}\label{sec:resultsPerPointBlue}
In \Cref{fig:bluePointsResult} it's shown that the blue group does get predicted most of the time for both one and four fingerprints/point.
It's also shown that the brown group is almost never predicted.
This can be explained by the location of the yellow group, which is closer than the brown group.
Between the area of the blue group and the yellow and brown there was a large concrete wall.
Beacons far away would be more affected by going through this wall than nearby beacons would, which is the case for the yellow group.

\fig{Measurement result per point for blue group}{bluePointsResult}{1}{results/pointsBlue}

In the cases where the yellow group is being predicted it can be explained by the beacon's placement.
At certain points beacons far away from the blue group, and closer to the yellow groups, has a clear line of sight.
A result from this is that nearby beacons have a weaker signal than beacons far away.
This explanation is true for both one and four points since the beacons has the same arrangement.

\bigskip

Why the brown group get predicted at all has its explanation in signal fluctuation.
If the signal of nearby beacons fluctuates a lot it will make a false prediction, that can be of a group further away.


\subsubsection{\mt{Yellow group}}\label{sec:resultsPerPointYellow}
In the yellow group the fluctuation is low with the majority calculated groups which is shown in \Cref{fig:yellowPointsResult}.
This is a result of a more open space where the signals are not being reflected by any walls, as well as more beacons are in range with a clear line of sight, which will give a better prediction accuracy.

\fig{Measurement result per point for yellow group}{yellowPointsResult}{1}{results/pointsYellow}

Since the brown and yellow group is close to each other minor fluctuations in the \acrshort{rssi} signals can cause the \acrshort{knn} classification to predict the wrong group in some cases, as seen in \Cref{fig:yellowPointsResult}.


\subsubsection{\mt{Brown group}}\label{sec:resultsPerPointBrown}
In the predicted points for the brown group presented in \Cref{fig:brownPointsResult} the result is not as self-explaining.
As seen in the figure, the blue group get predicted at two points.
This has to do with similar \acrshort{rssi} values of the beacons, where beacons near the blue group, and beacons in or near the brown group has similar values.
A result from this makes the prediction more unstable with the different mapped fingerprints in the groups.

\fig{Measurement result per point for brown group}{brownPointsResult}{1}{results/pointsBrown}

With one fingerprint/point the prediction is only correct in the first point.
This has the same explanation as in the beginning of this sub-section, that the \acrshort{rssi} values can fluctuate so much that the outcome of the prediction changes.
The prediction is stabilizing with four fingerprints/point because there is more data to predict against, and signal fluctuations doesn't have as large impact.


\subsection{System accuracy}\label{sec:resultsSystemAccuracy}
\Cref{fig:twoVsFour} presents the accuracy of the testing for each group with one fingerprint/point and four fingerprints/point, as well as the overall accuracy.

\fig{Total accuracy per group for one and four mappings}{twoVsFour}{1}{results/twoVsFour}

As seen in the right most part of \Cref{fig:twoVsFour}, the overall accuracy of four fingerprints (75\%) is higher than one fingerprint (44\%).
These differences are a result of how well the system can predict the location of the device.
With one fingerprint/point the position is sensitive to signal changes, since there is only one mapped fingerprint at each point in the created groups.
The \acrfull{knn} classification also has fewer fingerprints to test on which will make the prediction more uncertain.

\bigskip

With four fingerprints/point the accuracy is higher, since the prediction is not as sensitive to signal fluctuation because there are more fingerprints to test against.
This means that the \acrshort{knn} classification is able to make better predictions. 
