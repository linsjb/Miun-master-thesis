\section{Bluetooth Low Energy}\label{sec:theoryBle}
The wireless technology \acrfull{ble} is part of the core specifications of Bluetooth 4.0 and is also referred to as Bluetooth Smart.
\acrshort{ble} focuses on short-range communication where the transmission strength often is measured in dBm.
A closer distance between the receiver and transmitter means a higher dBm-value, which typically range between around -30 dBm to 0 dBm.
\cite{DevelopmentMobileIndoor2017} 

\bigskip

\mt{The \acrshort{ble} protocol is designed to send messages with a short duration.
With this comes a low-power consumption and often low cost, which makes \acrshort{ble} a good candidate for various different tasks.
\cite{PracticalFingerprintingLocalization2017, LocationFingerprintingBluetooth2015}}

\bigskip

\mt{WiFi and \acrshort{ble} is working in the same radio frequencies, but \acrshort{ble} does not suffers from the same problems as WiFi does.
WiFi has a passive scan, which lowers the update rate and make the positioning harder because of fewer updates.
Also, WiFi wasn't designed to broadcast the signal strength in any specific unit.
\cite{LocationFingerprintingBluetooth2015}}


\subsection{iBeacons}\label{sec:theoryBleiBeacons}
iBeacon are small battery powered devices developed by Apple that is based on the existing \acrshort{ble} technology.
Using the iBeacon technology together with the Apple iOS platform open up for a variety of opportunities when it comes to implementing position-based applications.
Because of the small footprint and low cost of these devices, they are easy to deploy in an environment where they can be used together with a smartphone application.
\cite{BluetoothLowEnergy2018} 

\bigskip

The iBeacon is sending out an advertisement signal that consists of three parts \cite{GettingStartedIBeacon2014}.
\begin{itemize}
\item 16-byte UUID
\item 2-byte major value
\item 2-byte minor value
\end{itemize}

The UUID value is a fixed identifier that will identify a set of beacons.
This value needs to be the same for all the beacons that will be used.
The major value is used to identify a large set of beacons with the same UUID, and the minor value is used to identify a specific beacon.
\cite{GettingStartedIBeacon2014}

\bigskip

An example of the iBeacon deployment mentioned in \cite{GettingStartedIBeacon2014} is a global retail store. The UUID will be the same for all stores, the major value is used to identify a specific store and the minor value to identify a department in each of the stores.
\cite{GettingStartedIBeacon2014} 

