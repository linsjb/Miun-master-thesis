\section{Positioning fingerprint}\label{sec:theoryFingerprint}
Fingerprints is a popular method to use in positioning systems that does not rely on GPS-signals.
These fingerprints are often based on \acrshort{rssi} values collected from the transmitters used in the \acrfull{ips}.
\cite{LocationFingerprintingInfrastructure2004, AutomaticConstructionRadio2018}

\bigskip

\mt{However, a system that use fingerprints can run into problems.
It's shown that the collected \acrshort{rssi} values that form's the fingerprints can vary between different devices, even if the devices are of the exact same model.
\cite{WhyFingerprintbasedIndoor2014}}

\bigskip

Apart from \acrshort{rssi}, fingerprints can be built with \acrfull{pdr} traces.
\acrshort{pdr} uses the accelerometer and the compass in the mobile phone.
With this the number of walked steps can be counted from a starting point and then position the device.
\cite{AutomaticConstructionRadio2018, NoNeedWardrive2012}
\mt{With \acrshort{pdr} the system has a high error propagation, which is a result from relative movements.
Different step lengths and that each step often have a different direction will make the system to take approximations of the position.
To solve this, the \acrshort{ips} need to be recalibrated at a fixed set of intervals.
For a \acrshort{pdr} system to be more precise it needs to be combined with other techniques such as \acrshort{rssi} or regular GPS.
\cite{RSSIPDRBasedProbabilistic2018}}

\bigskip

Fingerprint positioning can be used both for indoor and outdoor positioning.
In an \acrshort{ips} WiFi \acrshort{ap} transmitters is the most popular alternative.
\cite{LocationFingerprintingInfrastructure2004,
IndoorFingerprintPositioning2017}.

\bigskip

Fingerprint based localization is not only used in WiFi based systems but can
also be used with other hardware, such as \acrlong{ble} Beacons.
\cite{PracticalFingerprintingLocalization2017} 

\bigskip

A fingerprint positioning system is built up in a two-phase process.
An offline and online phase.\cite{IndoorFingerprintPositioning2017} 


\subsection{Offline phase}\label{sec:theoryFingerprintOffline}
In the offline phase (also called observation phase) the fingerprints are being collected and
mapped to a location in the environment.
The mapped fingerprints are then being stored in a radio map database.
\Cref{fig:fingerprintOfflinePhaseIllustration} shows the flow for collecting the \acrshort{rssi} signals, mapping them to a location and storing them in the database.
\cite{IndoorFingerprintPositioning2017} 

\fig{Fingerprint offline phase flow \cite{IndoorFingerprintPositioning2017}}{fingerprintOfflinePhaseIllustration}{0.9}{fingerprintOfflinePhase}


\subsection{Online phase}\label{sec:theoryFingerprintOnline} The online phase
is the part where a user is interacting with the stored fingerprints.
A device is collecting \acrshort{rssi} values from nearby senders in real-time.
These collected values is then being compared with the fingerprints in the database.
If a match between the real-time \acrshort{rssi} values and the radio map occur the location is being yield back and the device can be positioned.
\Cref{fig:fingerprintOnlinePhaseIllustration} show an illustration of the
online phase.
\cite{IndoorFingerprintPositioning2017}

\fig{Online phase positioning flow\cite{IndoorFingerprintPositioning2017}
}{fingerprintOnlinePhaseIllustration}{1.0}{fingerprintOnlinePhase}

