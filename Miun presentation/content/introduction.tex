\section{Introduction}


\begin{notedFrame}{Background}{intro/background.tex}
	\begin{columns}[T, onlytextwidth]
		\column{0.5 \textwidth}
			\begin{itemize}
				\item Case study at LKAB
				\item Has a current application
				\item Apple iOS
				\item OCR codes scanning
				\item Get data automatic
			\end{itemize}

		\column{0.5 \textwidth}
			\figNoCap{1.0}{lkabLogo}
	\end{columns}
\end{notedFrame}

\begin{notedFrame}{Overall aim}{intro/aim.tex}
    \begin{block}{}
    \begin{itemize}
        \item Investigate how indoor positioning works
        \item Implement a proof of concept application for LKAB's specific needs
    \end{itemize}
    \end{block}
\end{notedFrame}


\begin{notedFrame}{Problem statement}{intro/problem.tex}
\begin{exampleblock}{}
    How, as accurate as possible, locate handheld devices in indoor environments without any GPS signal?	
\end{exampleblock}
\end{notedFrame}


\begin{notedFrame}{Project goals}{intro/goals.tex}
	\begin{enumerate}
		\item
		Investigate how an Indoor Positioning System works by studying the field

		\item
		Design a system that can position a device without GPS-signals

		\item
		Develop a proof of concept Apple iOS application that implements the designed system

		\item
		Evaluate the performance of the implemented Indoor Positioning System by physical testing
	\end{enumerate}
\end{notedFrame}

\begin{notedFrame}{Understand Indoor Positioning System}{intro/understand.tex}
	\begin{enumerate}
	\item Read nearby transmitter signals
	\item	Map and store a localization point with the current signals
	\item Position the device by comparing real-time signals to the stored localization points
	\end{enumerate}
\end{notedFrame}